\documentclass[12pt, dvipsnames, a4paper]{article}
\usepackage{geometry}
\geometry{legalpaper, margin=0.5in}
\usepackage{xcolor}
\usepackage{lipsum,etoolbox}
\usepackage{xspace} 
\usepackage[normalem]{ulem}
\usepackage{vwcol}
\usepackage{cancel}
\usepackage{enumitem}
\usepackage{amsmath}
\usepackage{caption}
\usepackage{graphicx}
\usepackage{amsfonts}
\usepackage{float}
\usepackage{multicol}
\usepackage{hyperref}
\usepackage{listings}
\usepackage{textcomp}
\usepackage{lstautogobble}
\usepackage[parfill]{parskip}
\usepackage{tikz-qtree}
\usepackage{tikz}
\usepackage{hyperref}
\usepackage[notransparent]{svg}
\svgpath{{./assets/}}
\usetikzlibrary{decorations.pathreplacing}
\tikzset{every tree node/.style={minimum width=4cm,draw,circle},
         blank/.style={draw=none},
         edge from parent/.style=
         {draw,edge from parent path={(\tikzparentnode) -- (\tikzchildnode)}},
         level distance=1.5cm}

%% Genearl %%
\renewcommand{\thesection}{\arabic{section}}


%% For convenience %%
\newcommand{\code}[1]{\texttt{#1}}
\newcommand{\bcode}[1]{\texttt{\textbf{#1}}}
\newcommand{\balert}[1]{\textbf{\alert{#1}}}
\newcommand{\rarrow}{$\Rightarrow$}
\newcommand{\tab}[1][0.5cm]{\hspace*{#1}}
\newcommand{\deepemphasis}[1]{\underline{\textbf{\Large{#1}}}}
\newcommand{\bfemph}[1]{\textbf{\emph{#1}}}
\newcommand{\OR}[0]{\lvert \: \rvert}

%% Colours %%
\definecolor{mLightBrown}{HTML}{EB811B}
\definecolor{mLightGreen}{HTML}{14B03D}

%% Pseudocode %% 
\lstdefinelanguage{pseudo}
{
	keywords=[1]{
		let,
		class,
		new,
		loop,
		until,
		end,
		if,
		else,
		then,
		return,
		while,
		for,
		to,
		fun,
		break,
		and,
		true,
		false,
		or,
		do,
		max,
		min,
		elif,
	},
	keywordstyle=[1]\color{black}\bf,
	keywords=[2] {
		invariant,
		precond,
		postcond
	},
	keywordstyle=[2]\color{blue}\bf
}

\lstset{
	breaklines		=	true,
	language 		= 	pseudo,
	basicstyle		=	\ttfamily,
	mathescape		=	true,
	escapeinside	=	||,
	tabsize			=	2,
	numbers			=	left,
	commentstyle	=	\color{OliveGreen},
	stringstyle		=	\color{mLightBrown},
	upquote			=	true,
	morestring		=	[b]',
	moredelim		=	[l][\rmfamily\itshape]{@},
	comment			=	[l]{//},
	morecomment		=	[s]{/*}{*/},
	commentstyle=\color{Gray}\ttfamily,
	showstringspaces=	false,
	showtabs		=	false,
	autogobble
}

%% Other %%
\setcounter{secnumdepth}{5}
\setcounter{tocdepth}{5}

% \patchcmd{<cmd>}{<search>}{<replace>}{<success>}{<failure>}
\patchcmd{\abstract}{\titlepage}{\titlepage% Insert ToC-writing after starting a titlepage
  \addcontentsline{toc}{chapter}{Abstract}}{}{}
\setcounter{secnumdepth}{3}
\setcounter{tocdepth}{3}

% Keywords command
\providecommand{\keywords}[1]
{
  \small	
  \textbf{\textit{Keywords---}} #1
}


%**************************************************************************************************************%
%______________________________________________________________________________________________________________%
\begin{document}
\title{\textbf{EECS 4314 - Bit Theory\\Architecture Report}}
\date{\Large \today}
\author{
	\large \textbf{Amir Mohamad}\\ \small amohamad@my.yorku.ca\\\\
	\large \textbf{Arian Mohamad Hosaini}\\ \small mohama23@my.yorku.ca\\\\
	\large \textbf{Dante Laviolette}\\ \small dantelav@my.yorku.ca\\\\
	\large \textbf{Diego Santosuosso Salerno}\\ \small nicodemo@my.yorku.ca\\\\
	\large \textbf{Isaiah Linares}\\ \small isaiah88@my.yorku.ca\\\\
	\large \textbf{Joel Fagen}\\ \small joefagan@my.yorku.ca\\\\
	\large \textbf{Misato Shimizu}\\ \small misato1@my.yorku.ca\\\\
	\large \textbf{Muhammad Hassan}\\ \small furquanh@my.yorku.ca\\\\
	\large \textbf{Yi Qin}\\ \small aidenqin@my.yorku.ca\\\\
	\large \textbf{Zhilong Lin}\\ \small lzl1114@my.yorku.ca\\\\
	\large York University\\
}
\maketitle
\newpage
\hspace{0pt}
\vfill
\begin{abstract}
	\lipsum[1]
	\lipsum[1]
	\\\\
	\keywords{keyword1, keyword2, keyword3}
\end{abstract}
\vfill
\hspace{0pt}
\newpage
\tableofcontents
\clearpage

\section{Introduction}
\lipsum[1]

\subsection{Overview}
\lipsum[1]

\section{Architecture}


What is the control and data flow among parts?\par
Data flow is how data flows among system components during processes. Control flow is the order of executions among components. The explanations below show how both flows are structured in subsystems of FreeBSD.\par
\textbf{Initialization}: In the initialization process, Basic input/output system (BIOS) with read-only memory (ROM) will be the first program appears in the flow. Under the control of BIOS, the processer finds the address \code{0xfffffff0} which contains the how-to of POST routines for low-level initializations. Then, as the last POST routine, BIOS follows an instruction, called \code{INT 0x19}, to reach out the address \code{0x7c00} and find a boot0 program. At the point when a boot0 program executes, the control will be taken over by the FreeBSD operating system. After taking over its control, the system will jump to multiple addresses to go through other boot programs and BTX server (boot1, BTX server, and boot2), and eventually reach a loader, which boots the system’s kernel.\par
\textbf{Inter-Process Communication}: The major component for IPC in FreeBSD is a descriptor file, called BSD sockets. For example, to obtain an object on a web page, the system recognizes the object, which is encoded as a file, and utilizes a HTTP protocol to obtain more information on the object. These processes can be completed in the system’s kernel. Then, BSD sockets take care of the process of utilizing Transmission Control Protocol (TCP) and Internet Protocol (IP) to obtain the object from a server.\par
\textbf{File System}: For the processes of communicating with devices, Zettabyte file system (ZFS) takes a significant role in the system. ZFS is a system that handles general file system jobs as well as the volume management jobs, and it is under the control of the operating system. By combining multiple devices into a pool, it generates a large file system. Based on this file system, data can be exchanged between a hardware device and programs. For programs to obtain data, stored in a device, an ISA driver controls the communication between them, and it copies the necessary data from the device memory into the main memory. Moreover, if data needs to be stored in a device, the driver takes the data address and sends it to the device. Then, the hardware’s Direct Memory Access (DMA) mechanism allows the device to access the data.\par 
\textbf{Memory Manager}: We can find a lot of components that the system’s memory manager handles. One of the responsibilities is kernel memory allocation, which the system utilizes virtual memory (VM) to handle page tables with addresses. For instance, system calls such as \code{brk} can be made by library functions (e.g., \code{malloc}) to control the allocation volume of a process, and the system’s kernel in the OS reach out to the main memory if necessary. The whole process here can be completed by the kernel in the system.\par


\section{Diagrams}
\lipsum[1]

\section{External Interfaces}
\lipsum[1]

\section{Use Cases}
\lipsum[1]

\section{Data Dictionary}
\lipsum[1]

\section{Naming Conventions}
\lipsum[1]

\section{Conclusion}
\lipsum[1]

\section{Lessons Learned}
\lipsum[1]

\begin{thebibliography}{00}
	\bibitem{b2} Clarke, Arthur C. 2001: A Space Odyssey. New York: Roc, 1968. 297.
\end{thebibliography}
\end{document}
